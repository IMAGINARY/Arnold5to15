\documentclass[10pt,a5paper,twoside]{article}
\usepackage[top=12mm,bottom=26mm,outer=28mm,inner=14mm,foot=14mm]{geometry}
\usepackage{calc}
\usepackage{scrextend}
\deffootnote[1.5em]{0em}{1em}{\thefootnotemark\quad}
\renewcommand{\footnoterule}{%
  \kern -2.4pt
  \hrule width \textwidth height 0.4pt
  \kern 2pt
}

\usepackage{fontspec}
\setmainfont[
	Ligatures=TeX,
	Extension=.otf,
	SlantedFont=cmunsl,
	BoldFont=cmunbx,
	ItalicFont=cmunti,
	BoldItalicFont=cmunbi,
	SmallCapsFont=cmunrm, % for upright instead of slanted small caps
	SmallCapsFeatures={Letters=SmallCaps,Numbers=OldStyle},	
]{cmunrm}

\usepackage{etoolbox}
\usepackage{microtype,ellipsis}

\usepackage{polyglossia}
\setotherlanguage{russian} % the name of the original Russian version at the end of this book is written using Cyrillic letters

\usepackage{textcomp}

\usepackage{amsmath,amssymb,nicefrac,amscd}
\usepackage{graphicx,float}
\usepackage{pdfpages}

\usepackage{enumitem}
\setitemize[1]{noitemsep,nosep,leftmargin=0.99em,label={--}}

\usepackage{transparent}
\usepackage{csquotes}
\DeclareQuoteStyle{vietnamese}% verified
  {\textquotedblleft}
  {\textquotedblright}
  [0.05em]
  {\textquoteleft}
  {\textquoteright}


\usepackage{siunitx}
\sisetup{per-mode=fraction,fraction-function=\nicefrac}

\usepackage{hyperref}

\usepackage{todonotes}

\newcommand{\eps}{\varepsilon}

% Usually, you would define a theorem-like enviroment which uses automatic numbering
% but Arnold also uses special numbering for some problems. Therefore, I kept the manual numbering.
\newenvironment{problem}[1]{\paragraph*{#1}}{}

\newenvironment{note}[1]{\par\noindent\textsc{\MakeLowercase{#1}} }{\par}

\makeatletter

% do no indent the first paragraph of the abstract
\let\oldabstract\abstract
\def\abstract{\oldabstract\noindent\@ifnextchar\par{\expandafter\abstract\@gobble}{}}

% always center contents of floats
\g@addto@macro\@floatboxreset{\centering}

% make all figures use 'H' position by default:
\def\fps@figure{H}
\makeatother
