\documentclass[10pt,a5paper,twoside]{article}
\usepackage[top=12mm,bottom=26mm,outer=28mm,inner=14mm,foot=14mm]{geometry}
\usepackage{calc}
\usepackage{scrextend}
\deffootnote[1.5em]{0em}{1em}{\thefootnotemark\quad}
\renewcommand{\footnoterule}{%
  \kern -2.4pt
  \hrule width \textwidth height 0.4pt
  \kern 2pt
}

\usepackage{fontspec}
\setmainfont[
	Ligatures=TeX,
	Extension=.otf,
	SlantedFont=cmunsl,
	BoldFont=cmunbx,
	ItalicFont=cmunti,
	BoldItalicFont=cmunbi,
	SmallCapsFont=cmunrm, % for upright instead of slanted small caps
	SmallCapsFeatures={Letters=SmallCaps},	
]{cmunrm}

\usepackage{microtype,ellipsis}

\usepackage{polyglossia}
\setotherlanguage{russian} % the name of the original Russian version at the end of this book is written using Cyrillic letters

\usepackage{textcomp}

\usepackage{amsmath,amssymb,nicefrac,amscd}
\usepackage{graphicx,float}
\usepackage{pdfpages}

\usepackage{enumitem}
\setitemize[1]{noitemsep,nosep,leftmargin=0.99em,label={--}}

\usepackage{transparent}
\usepackage{csquotes}
\usepackage{siunitx}
\sisetup{per-mode=fraction,fraction-function=\nicefrac}
\DeclareSIUnit[number-unit-product=\,]\uhr{Uhr}
\DeclareSIUnit[number-unit-product=\,]\zoll{Zoll}

\usepackage{hyperref}

\usepackage{todonotes}

\newcommand{\eps}{\varepsilon}
\newenvironment{problem}[1]{\paragraph*{#1}}{}
\newenvironment{note}[1]{\par\noindent\textsc{#1} }{\par}
